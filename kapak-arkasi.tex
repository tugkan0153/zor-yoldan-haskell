\thispagestyle{empty}
\fbox{
  \begin{minipage}{365pt} %

    \begin{center}
      \addvspace{180pt}
                {\fontsize{60pt}{70pt}\bf\selectfont\AabcedB\color{black!10} Zor Yoldan Haskell}
    \end{center}
    
    \addvspace{130pt}

\renewcommand{\arraystretch}{1.0}
\begin{threeparttable}\footnotesize
\begin{tabular}{rl}
Orjinal belge adı, çıkış       & Haskell the Hard Way, 8.Şubat.2012\\ 
Orjinal belge yazarı           & \yazar, Fransa\\ 
Türkçeye çeviren               & Joomy Korkut, Amerika, 2.Mayıs.2014\\
PDF'ye dönüştüren (\LaTeX ile) & Tuğkan-0153, \today \\ 
\end{tabular}
\end{threeparttable}
\vspace{2pt}

    \vspace{1pt}
    
   \begin{minipage}{0.96\textwidth}
      \begin{flushleft}
        \noindent{\footnotesize\color{blue}Ürün Düzenleme Payı \copyright\ 2012, \yazar;
        2014, Joomy Korkut; 2022, Tuğkan-0153\\
        \noindent\footnotesize\color{black!75}\hyperref[lisansstil]{GNU Özgür Dokümantasyon Lisansı} Versiyon 1.2 kapsamında bu belgeyi kopyalama, dağıtma de/yada düzenleme onayı verilmiştir.\\
        Belgenin kaynak kodları \texttt{github.com/tugkan0153} 'da yayınlanacaktır.}
\end{flushleft}
\end{minipage}
\end{minipage}
}

\vspace{1pt}

\noindent{\small Belgedeki linklerde son benek (örneğin \centerdot com 'da olduğu gibi) grafik benektir. Apple iBook gibi kimi PDF-göstericiler, PDF üzerindeki yazıları tarayarak ``.com'' ile biten yazıları, kliklenebilir olarak düzenlemiş olsalar bile kliklenebilir yapmaktadır. O durumda bilgisayar yada tabletinizde PDF'yi okurken yanlışlıkla link açmaktadır. Bu durumu engellemek için linklerde benek grafikleri kullanılmıştır. Linkleri açmak istediğinizde linki kopyalayıp grafik beneği normal benekle değiştirerek tarayıcıda açabilirsiniz.}

\vspace{1pt}

\noindent{\color{orange} ÖNEMLİ: \small Bu belge ana dili Fransızca olan yazar aracılığı ile yazılmış, Amerika'da yaşayan bir Türk aracılığı ile \texttt{Markdown} biçemi kullanılarak Türkçeye çevrilmiş, \today\ gününde elinizde gördüğünüz PDF formatına dönüştürülmüştür. PDF dönüşümünü yapan kişi olan ben Tuğkan, Haskell bilmemekteyim (\today\ gününü temel alarak) Dolayısıyla belgedeki kodların doğruluğunu garantileyemem, anlatı bölümlerinin eksiksiz doğru olduğunu öne süremem. BELGE GEREK KODLAR GEREK ANLATI BÖLÜMLERİNDE BİR YADA BİRÇOK YANIL İÇEREBİLİR}
